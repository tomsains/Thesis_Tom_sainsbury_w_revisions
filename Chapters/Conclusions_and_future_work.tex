\section{Summary}
In this thesis I investigated the role of visual experience in the development of population activity and visually guided behaviour using zebrafish as a model organism. In particular the effect of EE on hunting performance and developing spontaneous activity were examined. The main findings and developments of this thesis were:
\begin{enumerate}
    \item GR fish consume more prey than NR fish when hunting within the gravel environment but hunt the same amount as NR fish when hunting within the WB.  This environment specific effect suggests that hunting in GR fish is positively modulated by the context of the GB (\textbf{Chapter 4}).
    
    \item Spontaneous activity in the tecta of GR fish relative to NR fish was found to be altered in two ways: a) Single neurons in GR fish were found to have higher response amplitude, duration and firing frequency across multiple stages of development when compared to NR fish. b) Whilst size, firing frequency and synchrony of tectal assemblies were found to be similar between the two rearing conditions in the early stages of development (3-5 dpf), they followed a divergent development trajectory in the later stages (5-7 dpf). As a result, tectal assemblies at 7 dpf were larger, more active and less synchronous in the GR fish.  This shows that the development of the tectal activity is altered by EE (\textbf{Chapter 4}).
    
    \item All experience-dependent changes in the tectum are blocked in a zebrafish line that is a null for both genes encoding the NR2A subunit of the NMDAR. This indicates that the NR2A subunit receptor is necessary for EE to shape tectal population activity, and therefore that Hebbian forms of plasticity may mediate the effects of EE on functional development of the tectum (\textbf{Chapter 4}).
    
    \item A virtual reality hunting assay was developed to understand how the sensory representation of prey-like stimuli changed when these stimuli are viewed over textured/non-textured backgrounds. This was achieved by decoding the stimulus location from tectal activity. Preliminary results show that decoding performance is better when viewing stimuli over a textured background, suggesting that the tectal population response may be modulated by the visual context in which prey-like stimuli are presented (\textbf{Chapter 5}).
\end{enumerate}


Overall, this shows that EE shapes both visually guided behaviour and developing tectal activity in a process that may be dependent on the subunit composition of the NMDA receptor. In addition, it provides an experimental protocol that could be used to relate the effects of EE on behaviour to changes in the tectum in the future.


\section{The importance of natural visual features during development.}
Many approaches studying experience-dependent plasticity in the visual system have attempted to use manipulations that either restrict or reduce the visual information that an organism experiences, relative to normal lab rearing conditions. EE takes the opposite approach by enhancing the sensory experience. In this study placing gravel under the petri dish of developing embryos creates a purely visual EE which uses some natural features that would be present in a natural zebrafish habitat (\cite{Engeszer2007ZebrafishField}). Surprisingly, this subtle manipulation had profound effects on both hunting performance and developing population activity. 

One finding highlighted by the results of this thesis is that the normal laboratory rearing conditions for organisms are likely to be highly impoverished in terms of visual experience. This is something that has been well appreciated in rodent research for a number of years and as a result many labs have incorporated some form of enrichment into their standard rearing protocol (\cite{Bayne2018EnvironmentalPerspectives}). However, zebrafish embryos are still typically placed on a shelf in an incubator which is devoid of any natural features.  
As GR fish were able to consume more prey relative to NR fish it suggests that the normal rearing conditions may be detrimental to visual system development. This has important implications for those studying visual/behavioural neuroscience in zebrafish. This is because studying NR fish may underestimate the types of computation or behavior that the zebrafish visual system is capable of generating in the wild. Furthermore, it is possible that subtle changes in visual rearing conditions, between labs, could produce variability in behaviour and the development of visual system properties. 

The enrichment protocol presented in this thesis was designed to mimic certain features of the natural world by bringing them into the lab. However, it is obvious that this manipulation falls short of delivering the rich visual experience that zebrafish would experience in the wild. Therefore, it is possible that other naturalistic features, such as vegetation or other fish, may also impact visual system development. In addition to this, it is not yet clear from these experiments what features of the gravel are important for shaping the developing brain. For example, is it the colour, the differences in contrast or the different spatial frequencies that are important for its effects on the developing visual system? Or could it be a combination of them all? Future experiments could attempt to isolate which features of the gavel are important for driving the plastic changes observed in this thesis. For example, instead of using actual gravel, a printed picture of gravel could be used instead. This would enable certain features of the background to be manipulated, such as filtering out certain spatial frequencies, colors or changing the contrast in the image. Zebrafish larvae could then be raised over this modified gravel environment so that their neural activity could be imaged and their hunting assessed. If removing a particular visual feature alone is able to block the plastic changes that are usually associated with exposure to gravel this could indicate that that feature is being used to shape the developing visual system and behaviour. Alternatively, it may be that removing any one of  multiple visual features can block the plastic changes in the larvaes behaviour and tectal activity, suggesting that many different features in combination may be critical for inducing the plastic changes seen in this thesis. Furthermore, a major claim of this thesis is that it may be the structure of natural scenes specifically that are important for shaping the behaviour and the developing visual system of zebrafish larvae. Therefore it would be interesting to understand whether it is specifically the structure of natural scenes, or any textured background, that can induce the plastic changes seen in this thesis. To do this, zebrafish larvae could be exposed to a modified version of the gravel where the alignment between different lines in the image has been shifted, creating a phase scrambled version of the gravel background which lacks the spatial structure contained in the original image (\cite{Guo2005Centre-surroundCortex, Pecka2014Experience-DependentScenes}).  

\section{Visually guided behaviour is shaped by the visual environment.}
EE enrichment was found to have a major, yet unexpected, effect on the number of prey that could be consumed by GR fish. This is because GR fish were found to consume more prey that normally reared fish exclusively when they were hunting over the GB. This shows that EE isn't simply accelerating/enhancing visual system development. Instead it suggests that the fish may be using either a specific visual feature, or the combination of many features contained within the background, to their advantage. In the wild, where prey would typically be viewed against a complex background, this could provide a behavioural advantage that promotes survival. Since this thesis has only looked at the number of prey consumed it is not yet clear what aspect of hunting is being altered to facilitate this change. However, the presence of gravel during development may modify the visual system to either enhance the ability of fish to detect or locate prey. It is also possible, yet unlikely, that the presence of of the GB is able to exclusively increase the motivational state of GR fish, resulting in more prey being consumed.

To understand which specific aspect of the behaviour is modified, future experiments could aim to provide a more complete description of the differences in hunting sequences by tracking the tracking GR and NR fish while they are hunting in the two visual environments. In zebrafish, hunting sequences begin with eye convergence and the eyes remain converged for the duration of the hunting sequence (\cite{Bianco2011}). Whilst eye convergence can occur in the absence of prey, these events are extremely rare and almost never exceed a convergence angle of 60$^{\circ}$. Therefore, sustained eye convergence exceeding 60$^{\circ}$ can be used as a measure of hunting event frequency or rate (\cite{Lagogiannis2019LearningLarvae}). An increase in hunt rate in GR fish could be indicative of either a increase in motivational state or in the ability to detect prey. However, if these hunt events in GR fish are initiated when the fish is further away from the prey, it could suggest that the change in hunt frequency is driven by changes in perception rather than motivation. In addition to hunt rate, the heading direction of the fish during hunt sequences relative to the prey could also be examined (\cite{Avitan2019}). This could be used to calculate how accurate the turns in the hunting sequence are. More accurate turns may indicate that the fish is better at locating the prey in visual space, leading to more efficient hunt sequences through on-target turns. Alongside this it may also be expected that the duration of time from the beginning of the hunt sequence to the capture may also be reduced, allowing the fish to hunt more prey within a certain time frame. It may be interesting to do these experiments in petri dishes where both background are contained within the same petri dish ie. one half has a WB and the other has GB. This would allow for the differences in behaviour to be visualised within the same fish as it swims over the two different backgrounds. Any observed changes are likely to reflect differences in the perception of the prey against the background rather that the modulation of motivational state which tend to occur over longer timescales.

Any behavioural changes associated with a change in detection or localisation of the prey could be caused by a contextual modulation of tectal responses which sharpens the perception of prey size, location, contrast and/or direction of motion.  This is plausible as similar modulation of neural responses have been seen in various species (\cite{Krause2014ContextualCortex, Sun2002ContextualPigeons, Sun2006ONRetina, Ziemba2018ContextualV2, Pecka2014Experience-DependentScenes}). One way to test if this is occurring is to allow the zebrafish larvae to hunt artificial prey that are projected onto the side of the petri dish and to image both the heading direction and the eyes. By using artificial prey, it would be possible to modify attributes of the prey such as the contrast, speed and size of the prey when it is presented over both gravel and non-gravel backgrounds. Any difference between GR and NR fish in these behavioural parameters would provide  evidence that the perception of the prey is contextually modulated by the presence of the gravel background rather than the motivation of the fish. However, it would also reveal specifically what attributes in perceiving the prey are being altered to cause the behavioural change.

While the focus of this thesis has been to look at the effect of EE on prey capture, what has not been addressed here is the potential effect on other visually guided behaviours. This may be important because within these early stages of development, in parallel to prey capture, zebrafish are developing a rich repertoire of visually guided behaviours such as the OKR, OMR and predator avoidance (\cite{Naumann2016FromResponse, Portugues2014, Dunn2016}). It is possible that the performance of these behaviours may also be altered by EE and would indicate that the visual environment can play diverse roles on many aspects of developing visuomotor behaviour. 

\section{What does spontaneous activity represent?}
A major component of this work involved studying the effect of EE on the development of spontaneous activity. The rationale for doing this is that correlated activity is likely to be seen between neurons that are connected to one another. Consistent with this, spontaneous activity has been seen to reflect the structure of visual evoked activity in multiple species, including zebrafish (\cite{Miller2014}; \cite{Romano2015}; \cite{Luczak2007}; \cite{Kenet2003}). Therefore studying the spatiotemporal structure of spontaneous activity can act as a proxy for understanding changes in connectivity within the tectum (\cite{Marachlian2018PrinciplesTectum}). In this thesis, investigating how the organisation of spontaneous activity changed throughout development demonstrated that the tectal assemblies of GR follow a completely divergent developmental trajectory between 5 - 7 dpf with them containing more neurons, being less synchronous and more active. This indicates that visual experience can have a profound effect on developing tectal circuitry. 

One disadvantage of looking at the development of tectal assemblies is that it is not clear how changes in their structure and dynamics relate to the changes that are also observed in prey consumption.  Spontaneously active tectal assemblies have been shown to exhibit substantial overlap with those that are visually evoked by prey-like stimuli and correlate with spontaneous J-turns (\cite{Romano2015}). It is not yet known what computations may be performed by the combined activity of neurons within an assembly. As a result, it is difficult to speculate how an increased number of neurons or decreased synchrony within an assembly might contribute to increased prey consumption. Therefore, it is necessary to understand what the functional composition of neural assemblies is more fully and how this composition changes with exposure to gravel. Currently it is known that tectal assemblies consist of neurons with different functional properties with some neurons being tuned to different directions whereas others have been found to have mixed selectively for speed, size and contrast that would be expected from moving prey-like object (\cite{Romano2015, Bianco2015}). One possibility is that the GB provides a noisy environment which makes robust decoding of the presence of prey more challenging. As a result, neurons of a similar functional sub-type to those present in the tectal assemblies of NR fish may be added to the tectum to provide redundancy, leading to more robust detection of prey and activation of downstream motor targets. Alternatively, the presence of gravel during development may lead to the incorporation of different functional subtypes into the assembly, potentially altering the computation that the assembly is performing all together. This could be important for GR fish to learn how to segment salient visual features such as prey from the background in natural visual scenes. This could be achieved through a type of contextual modulation of the neuronal responses called "iso-feature" suppression. This is where neurons responding to similar features of the background inhibit each other through GABAergic inter-neurons, suppressing each other and and this leads to a sharpening the tuning of neurons responding to the salient feature (\cite{Zhaoping2016FromNeuroscience, Zhaoping2012PropertiesBehavior, Pecka2014Experience-DependentScenes}). 

To investigate how the functional composition of neural assemblies may be altered by the presence of gravel, spontaneously activie neural assemblies could be mapped and then fish could be presented with a stimulus barrage while neural activity is recorded from the tectum (Shallcross and Diana, Unpublished). This stimulus barrage would consist of a range of different stimuli that are	known to elicit	responses from different functional	subtypes of	neurons. Clustering the resulting neural responses and comparing the resulting maps to those of spontaneously active neural assemblies could reveal the functional composition of assemblies. Furthermore, doing these experiment in a zebrafish line where GABAergic populations of neurons are labeled in red would allow for both the functional subtype of neurons to be identified and their neurotransmitter identity. Mapping neural assemblies in this way would give a better idea of the computations that may be performed by tectal assemblies and how those computations change with EE during development.

Whilst spontaneous activity has been used in this thesis as proxy for functional connectivity one question that has not yet been fully discussed is why does spontaneous activity even exist in the first place? This is an interesting question because spontaneous activity consumes a huge amount of the brain's total energy budget, making it energetically expensive (\cite{Tomasi2013EnergeticConnectivity}). This suggests that spontaneous activity is likely to serve a role that is important for survival. This could be for a number of reasons. Firstly,  spontaneous activity could be generated by an increase in the gain of the visual system in response to low levels of light. This could make neurons more likely to fire visual stimuli, potentially enhancing the ability of the animal to perceive objects in low levels of light. Secondly,  it could be related to the ongoing generation of motor actions, either participating in generating them or acting as efference copies (\cite{Stringer2019SpontaneousActivity}). Thirdly, spontaneous activity in the visual system may be important for consolidating the transient effects of past sensory experience into circuit level changes or contributing to short-term memory (\cite{Han2008}). Finally, one very intriguing suggestion is that spontaneous activity encodes an internal model of the sensory world (or a Bayesian prior over sensory experience) (\cite{Rich2015NeuralInference}). Consistent with this, one study in the cortex of ferrets has shown that over the course of development the statistics of spontaneous activity become matched to those of visually evoked activity caused by natural images but not artificial stimuli (\cite{Berkes2011}). This suggests that spontaneous activity is encoding the statistics of previously experienced visual scenes. The fact that the spatiotemporal structure of tectal activity is shaped by the environment may also be in-line with this. As a result it is possible that rather than being related to prey capture, the change in spontaneous activity reflects the typical visually evoked patterns of activity induced by EE. Therefore it would be interesting in future experiments, to understand if the patterns of spontaneous activity in GR or NR fish more closely resembles the patterns of visually evoked activity when viewing stimuli over a WB vs a GB.  This would reveal if spontaneous activity is able to reflect the statistics of visually evoked activity caused by the visual scene during development.

\section{Relating changes in tectal activity to changes in hunting performance}
The inherent difficulty in relating spontaneous activity in the tectum to prey consumption can be overcome by recording visually evoked activity from the tectum while the fish is behaving. To this end, in the final results chapter of this thesis, a virtual reality hunting assay was developed, allowing for simultaneous imaging of neural activity and behaviour while prey-like stimuli are presented to the fish. One potential reason for GR larvae consuming more prey is that they are able to use the textured background to better locate the prey in space. This can be tested within this setup by decoding the location of the prey-like stimulus from neural activity in the tectum when the stimuli are presented over textured/non-textured backgrounds. Preliminary results from these experiments suggest that the tectal response is modulated by the presence of the textured background, leading to improved encoding of stimulus location. One possible explanation for this is that the tectal responses are modulated by the context of the background to enhance the detection or localisation of prey in natural visual scenes. Such contextual modulation has been well described before with individual neurons having their responses modulated by different directions of motion, speed of motion and luminance in the background (\cite{Krause2014ContextualCortex, Sun2002ContextualPigeons, Sun2006ONRetina}). Furthermore, the structure of natural visual stimuli in the background has been shown to modulate the neural responses far more than phase scrambled versions of that same background (\cite{Guo2005Centre-surroundCortex}). This suggests that it is the spatial structure of natural scenes that is important for regulating contextual modulation and that contextual modulation may exist in order to perceptually segment salient stimuli from the background. 

In future experiments it would be interesting to understand what aspect of the perception of prey-like stimulus is actually changing to bring about the observed increase in decoding performance. This could be achieved within the virtual hunting assay by modifying stimulus attributes other than location and subsequently examining how decoding performance changes with the context of the background.  For example, prey-like stimuli could move at a range of different speeds, sizes and contrasts. Increasing the decoding performance of these attributes, when viewed over the gravel background could lead to a fish that is better able to identify prey in natural visual scenes. Additionally, tuning curves for individual neurons could be visualised in order to understand if the tuning of tectal neurons to these features are alter, potentially accounting of any change in decoding performance. 

In mice contextual modulation, such as center surround suppression, is learnt through sensory experience because it does not develop is mice have been dark reared (\cite{Pecka2014Experience-DependentScenes}).  Based on the behavioural result in this thesis where GR fish eat more prey exclusively when they are over the GB it was expected that contextual modulation would only occur in the GR fish. This would suggest that contextual modulation in zebrafish may also be learnt through exposure to natural visual scenes. At odds with this hypothesis, contextual modulation was seen in both the GR fish and the single NR fish suggesting that the two visual environments have no effect on the development of contextual modulation in the tectum. While it is possible that this single NR is an anomaly it could mean that, unlike mice, the tectum of zebrafish is hard wired to perpetually segment prey from the rest of a natural visual scene. This may not be a surprising result because of the rapid development of zebrafish with beginning to self feed at just 5 dpf. Therefore perhaps tectal circuits need to be capable of this type of computation much earlier in order to survive. It is also possible that this property may be slightly refined by experience as decoding performance was still highest in the GR fish overall. To really investigate if this is a hardwired feature of the tectum it would be necessary to see if this contextual effect also occurs in dark reared fish. This is because the NR fish still are exposed a some visual experience, although it is far more limited that the EE.  

Finally, it would be extremely useful in future experiments to perform both the imaging experiments (described in this section) and behavioural tracking experiments (described above) within the same fish. This is because the current results from the decoding assay do not yet provide a complete explanation for why the GR fish hunt more prey when hunting over the GB. Doing this would enable the decoding performance of the stimulus attributes such as speed, size and contrast and how they are modulated by the background to be correlated with a changes in the behaviour. This could be a powerful technique which has been used by others previously to predict behavioural performance from tectal activity throughout development (\cite{Avitan2019}) and would enable for the relationship between changes in neural activity, perception and behaviour to be fully realised.

In conclusion, this thesis to the best of my knowledge, provides the first evidence that both visually guided behaviour and developing population activity can be shaped by naturalistic features in the environment through EE. Although more work is needed to relate changes in population activity to behaviour, it provides an experimental protocol by which these questions can start to be answered in the future.


 
